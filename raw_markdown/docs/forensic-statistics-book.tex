% Options for packages loaded elsewhere
\PassOptionsToPackage{unicode}{hyperref}
\PassOptionsToPackage{hyphens}{url}
%
\documentclass[
]{book}
\usepackage{lmodern}
\usepackage{amssymb,amsmath}
\usepackage{ifxetex,ifluatex}
\ifnum 0\ifxetex 1\fi\ifluatex 1\fi=0 % if pdftex
  \usepackage[T1]{fontenc}
  \usepackage[utf8]{inputenc}
  \usepackage{textcomp} % provide euro and other symbols
\else % if luatex or xetex
  \usepackage{unicode-math}
  \defaultfontfeatures{Scale=MatchLowercase}
  \defaultfontfeatures[\rmfamily]{Ligatures=TeX,Scale=1}
\fi
% Use upquote if available, for straight quotes in verbatim environments
\IfFileExists{upquote.sty}{\usepackage{upquote}}{}
\IfFileExists{microtype.sty}{% use microtype if available
  \usepackage[]{microtype}
  \UseMicrotypeSet[protrusion]{basicmath} % disable protrusion for tt fonts
}{}
\makeatletter
\@ifundefined{KOMAClassName}{% if non-KOMA class
  \IfFileExists{parskip.sty}{%
    \usepackage{parskip}
  }{% else
    \setlength{\parindent}{0pt}
    \setlength{\parskip}{6pt plus 2pt minus 1pt}}
}{% if KOMA class
  \KOMAoptions{parskip=half}}
\makeatother
\usepackage{xcolor}
\IfFileExists{xurl.sty}{\usepackage{xurl}}{} % add URL line breaks if available
\IfFileExists{bookmark.sty}{\usepackage{bookmark}}{\usepackage{hyperref}}
\hypersetup{
  pdftitle={Forensic Science and Statistics: Version 1.0.0},
  pdfauthor={Jeff Holt and Joseph Swetonic},
  hidelinks,
  pdfcreator={LaTeX via pandoc}}
\urlstyle{same} % disable monospaced font for URLs
\usepackage{longtable,booktabs}
% Correct order of tables after \paragraph or \subparagraph
\usepackage{etoolbox}
\makeatletter
\patchcmd\longtable{\par}{\if@noskipsec\mbox{}\fi\par}{}{}
\makeatother
% Allow footnotes in longtable head/foot
\IfFileExists{footnotehyper.sty}{\usepackage{footnotehyper}}{\usepackage{footnote}}
\makesavenoteenv{longtable}
\usepackage{graphicx,grffile}
\makeatletter
\def\maxwidth{\ifdim\Gin@nat@width>\linewidth\linewidth\else\Gin@nat@width\fi}
\def\maxheight{\ifdim\Gin@nat@height>\textheight\textheight\else\Gin@nat@height\fi}
\makeatother
% Scale images if necessary, so that they will not overflow the page
% margins by default, and it is still possible to overwrite the defaults
% using explicit options in \includegraphics[width, height, ...]{}
\setkeys{Gin}{width=\maxwidth,height=\maxheight,keepaspectratio}
% Set default figure placement to htbp
\makeatletter
\def\fps@figure{htbp}
\makeatother
\setlength{\emergencystretch}{3em} % prevent overfull lines
\providecommand{\tightlist}{%
  \setlength{\itemsep}{0pt}\setlength{\parskip}{0pt}}
\setcounter{secnumdepth}{5}
\usepackage{booktabs}
\usepackage{amsthm}
\makeatletter
\def\thm@space@setup{%
  \thm@preskip=8pt plus 2pt minus 4pt
  \thm@postskip=\thm@preskip
}
\makeatother
\usepackage[]{natbib}
\bibliographystyle{apalike}

\title{Forensic Science and Statistics: Version 1.0.0}
\author{Jeff Holt and Joseph Swetonic}
\date{2022-01-26}

\begin{document}
\maketitle

{
\setcounter{tocdepth}{1}
\tableofcontents
}
\hypertarget{welcome}{%
\chapter{Welcome}\label{welcome}}

\textbf{This is a draft version of a work in progress. It is NOT intended for circulation.}

This book introduces statistical methods that are of use in forensic science.
In some cases the methods are currently used.
In others the methods are described but not generally in use,
although we hope that eventually they will be!

We assume that the reader has a basic understanding of mathematics, but no prior
knowledge of statistics is required.
We will provide a modest description of forensic methods, but as many of these are
highly nuanced we do not attempt to
provide a deep treatment of forensic methods here. However, when possible we will
try to provide references to more information.

There are portions of this book that borrow from the treatment of statistics
given in the online materials:

\begin{quote}
Online Statistics Education: A Multimedia Course of Study (\url{http://onlinestatbook.com/})
Project Leader: David M. Lane, Rice University
\end{quote}

This site provides a comprehensive introduction to statistics and is worth referencing when you would like additional information.
We deeply appreciate their willingness to allow the use of their work.

\hypertarget{an-introductory-example}{%
\section{An Introductory Example}\label{an-introductory-example}}

Consider the following scenario:
Suppose there is a late-night break-in at a closed convenience store.
No one is present to see the perpetrator, but there is low-quality CCTV footage.
Unfortunately, the \emph{only} thing that can be discerned from the video is the
color of the perpetrator's hair.

Now consider two possible versions of what comes next:

\textbf{Version 1}: The video shows that the perpetrator has \emph{brown} hair.
The day after the robbery, police see a person with brown hair walking down the street.
The person is arrested on suspicion of committing the break-in.

\textbf{Version 2}: The video shows that the perpetrator has \emph{green} hair. The
day after the robbery, police see a person with green hair walking down the street.
The person is arrested on suspicion of committing the break-in.

For the sake of this simplified example,
in both versions we assume that the color of the hair is the \textbf{only} reason the
police suspect the person who is arrested.

\textbf{Question}: In which version do you think it is more likely that the police have
arrested the person who committed the break-in?

Answering the question requires considering the likelihood that the police have
the correct person in both versions and deciding which is greater.
In most communities people with brown hair outnumber people with green hair,
so let's assume that is the case here. Intuitively,
it seems reasonable to expect that the likelihood of a correct arrest in Version 2 is greater
than that in Version 1:
Brown-haired people are relatively common, so in Version 1 the likelihood of
arresting the correct brown-haired person is probably low.
On the other hand, green-haired people are relatively rare, so in Version 2
the likelihood of arresting the correct person is higher.

Two important points:

\begin{enumerate}
\def\labelenumi{\arabic{enumi})}
\item
  Although the likelihood of the correct arrest is greater in Version 2
  than in Version 1 does not mean that the likelihood in Version 2 is high.
  It is possible that there are more brown-haired people than green-haired people
  while still having a lot of green-haird people, making the correct arrest
  unlikely in either case.
\item
  While the above discussion is interesting, the word ``likelihood'' is
  imprecise. Going forward we will develop ``probability'' which
  will allow us to quantify the notion of likelihood, so that we will be able
  to specify (for instance) how much more likely a correct arrest is in
  Version 1 than in Version 2.
\end{enumerate}

\hypertarget{introduction-to-probability}{%
\chapter{Introduction to Probability}\label{introduction-to-probability}}

\hypertarget{the-previous-example}{%
\section{The Previous Example}\label{the-previous-example}}

The example from the previous section involves the ``likelihood'' of arresting the
correct person based on their hair color. Here we begin the development of
``probability'' which allows us to quantify the notion of likelihood.

Let's start by providing some additional context for our example:
Suppose that the crime described takes place on an island of 1000 people.
The number of these people with each hair color is given in the table below.

\[
\begin{array}{c|c}
 \mathrm{Color} & \mathrm{Count} \\ \hline
\mathrm{Brown} & 670  \\ 
\mathrm{Blond} & 200  \\ 
\mathrm{Red} & 110  \\ 
\mathrm{Green} & 20  \\ 
\end{array}\\
\mbox{Table 1: Hair Color Counts}
\]

A \textbf{probability} is a number between 0 and 1 that provides a measure of the
likelihood that something occurs,
with a larger number indicating greater likelihood.

Suppose that we select a person from this population at random. To compute
the probability that this person has brown hair,
we take the number with brown hair and divide by the number of people:

\[\mbox{P(Brown hair)} = \frac{670}{1000} = 0.67\]

We use ``P(\ldots)'' to denote the probability of something. For instance in this case
\(\mbox{P(Brown hair)}\) = ``probability of brown hair''.

\hypertarget{sample-questions}{%
\subsection{Sample Questions}\label{sample-questions}}

Sprinkled thought this book are Sample Questions, which also play the role
of examples. In the online version, the
solutions to these questions are hidden. Before looking at the solution
(by hovering), we recommend that you try to do them yourself!

\textbf{1.} What is the probability that a randomly selected person has blond hair?

\textbf{Answer:} \(P(\text{Blond}) = \frac{200}{1000} = 0.20\).

\textbf{2.} What is the probability that a randomly selected person has red hair?

\textbf{Answer:} \(P(\text{Red}) = \frac{110}{1000} = 0.11\).

These examples are fine but do not answer the question from the previous section.
In Version 1, we have a brown-haired person on the CCTV and one of the
670 brown-haired people randomly arrested.
Thus the probability that the correct person is arrested is

\[\mbox{P(correct arrest)} = \frac{1}{670} = 0.00149\]

In Version 2 of the example we have a green-haired person on CCTV, so this
time the probability that the correct person is arrested is

\[\mbox{P(correct arrest)} = \frac{1}{20} = 0.05\]

Dividing the two probabilities

\[\frac{0.05}{0.00149} = 33.5\]

we see that a correct arrest in Version 2 is 33.5 times as likely as in Version 1.
However, even in Version 2 there is only a 0.05 probability that the police have
the correct person!

\hypertarget{sample-questions-1}{%
\subsection{Sample Questions}\label{sample-questions-1}}

\textbf{1.} Suppose CCTV shows a red-haired person committed the robbery, and that
a random red-haired person is arrested. What is the probability of a correct
arrest in this instance?

\textbf{Answer:} \(P(\text{correct arrest}) = \frac{1}{110} = 0.00909\).

\textbf{2.} Determine the number of times greater the likelihood of a correct arrest
if the culprit has red hair vs having blond hair.

\textbf{Answer:} In the case of a blond-haired person, we have
\(P(\text{correct arrest}) = \frac{1}{200} = 0.005\). Therefore the ratio
of probabilities is
\(\frac{0.00909}{0.005} = 1.81818\)
so a correct arrest for a red-haired culprit is 1.81818 times as likely
as for a blond-haired culprit.

\hypertarget{exercises}{%
\section{Exercises}\label{exercises}}

\begin{enumerate}
\def\labelenumi{\arabic{enumi}.}
\tightlist
\item
  \(\text{Put exercises here}\)
\end{enumerate}

\hypertarget{joint-probability}{%
\chapter{Joint Probability}\label{joint-probability}}

\hypertarget{blood-types}{%
\section{Blood Types}\label{blood-types}}

Blood plays a role in many forensic science applications, with the ``blood
type'' determined by a combination of two different blood group systems:

\begin{quote}
\emph{Blood types are inherited and represent contributions from both parents.
As of 2019, a total of 41 human blood group systems are recognized by the
International Society of Blood Transfusion (ISBT). The two most
important blood group systems are ABO and Rh; they determine someone's
blood type (A, B, AB, and O, with +, − or null denoting RhD status)
for suitability in blood transfusion.}

Source: \url{https://en.wikipedia.org/wiki/Blood_type}
\end{quote}

Suppose that we have a collection of 1000 people, with each classified based on both
ABO (A, B, AB, or O) and Rh (\(+\) or \(-\)).
The number of people of each combination of ABO and Rh class (the ``blood type'') is shown
in the table below. (This table is based on the distribution of blood types in
Canada, reported at \url{https://www.blood.ca/en/blood/donating-blood/whats-my-blood-type})

\[
\begin{array}{c|c|c}
       & + & - \\ \hline
\mathrm{A} & 360 & 60 \\ 
\mathrm{B} & 76 & 14 \\ 
\mathrm{AB} & 25 & 5 \\ 
\mathrm{O} & 390 & 70 \\ 
\end{array}\\
\mbox{Counts by ABO and Rh groups}
\]

Suppose that one of these people is selected at random and note that person's groups.
There are various possibilities for outcomes, such as the person selected has
ABO group O and Rh group \(+\).
The likelihood that this combination occurs is the \textbf{joint probability} of the two classifications,
ABO and Rh groupings.
There are 390 out of the 1000 people are O and \(+\), so the probability of randomly
selecting such a person is

\[\text{P(O and +)} = \frac{390}{1000} = 0.39\]

Here P(O and \(+\)) stands for the probability that the person selected is
ABO group O \textbf{and} Rh group \(+\).
A sample of other probabilities that come from the table are

\[\begin{array}{rcccl}
\text{P(AB and }-) & = & {\displaystyle\frac{5}{1000}} & = & 0.005 \\[5pt]
\text{P(B and +)} & = & {\displaystyle\frac{76}{1000}} & = & 0.076 \\[5pt]
\text{P(B and }-) & = & {\displaystyle\frac{14}{1000}} & = & 0.014 \\
\end{array}\]

We also can combine table entries to compute other probabilities.
For instance, there are a total of \(76 + 14 = 90\) people with blood group B,
so the probability of randomly selecting a person with blood group B is

\[
\mbox{P(B)} = \frac{76+14}{1000} = \frac{90}{1000} = 0.09
\]

As another example, the number of people with
Rh group \(+\) is \(360 + 76 + 25 + 390 = 851\). Thus the probability of randomly selecting
a person with Rh group \(+\) is

\[
\mbox{P(+)} = \frac{360 + 76 + 25 + 390}{1000} = \frac{851}{1000} = 0.851
\]

\hypertarget{sample-questions-2}{%
\subsection{Sample Questions}\label{sample-questions-2}}

\textbf{1.} Find the probability that a randomly selected person has blood
groups A and \(-\).

\textbf{Answer:} \(\text{P(A and }-) = \frac{60}{1000} = 0.06\).

\textbf{2.} Find the probability that a randomly selected person has ABO group AB.

\textbf{Answer:} There are \(25 + 5 = 30\) people that have ABO group AB, so that
\(\mbox{P(AB)} = \frac{30}{1000} = 0.03\).

\textbf{3.} Find the probability that a randomly selected person has ABO group A and
ABO group B and Rh group \(+\).

\textbf{Answer:} The is no person that has ABO group A \emph{and} ABO group B, so
\(P(\text{A and O and +}) = \frac{0}{1000} = 0\).

\begin{center}\rule{0.5\linewidth}{0.5pt}\end{center}

\hypertarget{and-vs-or}{%
\section{And vs Or}\label{and-vs-or}}

Above we saw that \(\text{P(B and }+) = 0.076\). Suppose we instead would like
to know \(\text{P(B or }+)\)?
That is, we want to know the probability that a randomly selected person
is either ABO group B \textbf{or} Rh group \(+\).
Note that this includes those people who are both ABO group B \textbf{and} Rh group \(+\).

From our table we see that the total number of people satisfying one or both
conditions is \(360 + 76 + 24 + 390 + 14 = 864\) so that

\[
\text{P(B or +)} = \frac{864}{1000} = 0.864
\]

Other combinations are also possible.
For instance, number of people with ABO group A blood or ABO group B blood
is
\[
\underbrace{(360 + 60)}_\text{ABO group A} + \underbrace{(76 + 14)}_\text{ABO group B} = 510
\]

Therefore we have

\[
\mbox{P(A or B)} = \frac{510}{1000} = 0.51
\]

\hypertarget{sample-questions-3}{%
\subsection{Sample Questions}\label{sample-questions-3}}

\textbf{1.} Find the probability that a randomly selected person has ABO group AB
or Rh group \(-\) blood.

\textbf{Answer:} The number of people who have ABO group AB \textbf{or} Rh group \(-\) is
\(25 + 5 + 60 + 14 + 70 = 174\) so that
\(\text{P(AB or }-) = \frac{174}{1000} = 0.174\)

\textbf{2.} Find the probability that a randomly selected person has ABO group O
or Rh group \(+\) blood.

\textbf{Answer:} The number of people who have ABO group O \textbf{or} Rh group \(+\) is
\(390 + 70 + 360 + 76 + 25 = 921\) so that
\(\text{P(O or }+) = \frac{921}{1000} = 0.921\)

\textbf{3.} Find the probability that a randomly selected person has ABO group AB or O.

\textbf{Answer:} The number of people who have ABO group AB \textbf{or} ABO group O is
\(25 + 5 + 390 + 70 = 490\) so that
\(\text{P(AB or O)} = \frac{490}{1000} = 0.49\)

\textbf{4.} Find the probability that a randomly selected person has ABO group A or AB,
and also Rh group \(-\).

\textbf{Answer:}
We have \(\text{(A or AB) and }(-) = (\text{A and }-) \text{ or } (\text{AB and }-)\).
The number of people who are \((\text{A and }-)\) is 60, and the number of
people who are \((\text{AB and }-)\) is 5, so that
\(P(\text{(A or AB) and }(-)) = \frac{60+5}{1000} = 0.065\).

\begin{center}\rule{0.5\linewidth}{0.5pt}\end{center}

\hypertarget{probability-tables}{%
\section{Probability Tables}\label{probability-tables}}

Frequently, tables provide probabilities (or percentages) instead of counts.
To make the conversion from counts to probabilities, all we do is divide each table entry by the total.
Thus, for our table, we divide each entry by 1000 to arrive at

\[
\begin{array}{c|c|c}
           & \mathrm{F} & \mathrm{M}\\ \hline
\mathrm{A} & 0.175 & 0.235 \\ 
\mathrm{AB} & 0.016 & 0.024 \\ 
\mathrm{B} & 0.037 & 0.063 \\ 
\mathrm{O} & 0.202 & 0.248 \\ 
\end{array}\\
\mbox{Gender and blood type counts}
\]

Table 2 entries give the probability of selecting each combination. For instance

\[
\mbox{P(Female and Type O)} = 0.202
\]

We can add entries in this table to compute other probabilities.\\
For example, suppose we want to compute the probability that a randomly selected person has type O blood.
Based on the table, this can happen if a person is female and has type O blood, or if a person is male and has type O blood.
Because these two groups are disjoint, we can add the probabilities to arrive at

\[
\mbox{P(Type O)} = \mbox{P(Female and Type O)} + \mbox{P(Male and Type O)}
= 0.202 + 0.248 = 0.45
\]

If instead (for instance), we want to compute P(Female) then we add the entries in the corresponding column, giving us

\[
\mbox{Pr(Female)} = 0.175 + 0.016 + 0.037 + 0.202 = 0.43 
\]

\begin{center}\rule{0.5\linewidth}{0.5pt}\end{center}

\hypertarget{exercises-1}{%
\section{Exercises}\label{exercises-1}}

\begin{enumerate}
\def\labelenumi{\arabic{enumi}.}
\tightlist
\item
  \(\text{Put exercises here}\)
\end{enumerate}

\hypertarget{conditional-probability}{%
\chapter{Conditional Probability}\label{conditional-probability}}

Now suppose that we just focus on the females in the population, which forms a subset of the population.
The probability that a randomly selected female has blood type O is an example of a \textbf{conditional probability}.

There are 430 females in the population, and 202 of those have type O blood.
Hence, the probability that female chosen at random has type O blood is

\[
\mbox{P(Type O | Female)} = \frac{202}{430} = 0.470
\]

The vertical bar in the notation is interpreted as \emph{given that}, so that \(\mbox{P(Type O | Female)}\) is read as

\[
\mbox{The probability of blood type O, given the person is female.}
\]

Conditional probabilities arise all the time when evaluating forensic evidence. Other examples of conditional probabilities:

The probability of Type AB, given that the person is male:

\[
\mbox{P(Type AB | Male)} = \frac{24}{570} = 0.042
\]

The probability the person is female, given Type B blood:

\[
\mbox{P(Female | Type B)} = \frac{37}{100} = 0.37
\]

The probability the person is male, given Type A blood:

\[
\mbox{P(Male | Type A)} = \frac{235}{410} = 0.573
\]

\hypertarget{sample-questions-4}{%
\subsection{Sample Questions}\label{sample-questions-4}}

\textbf{1.} What is the probability of blood type AB or B given the person selected is female?

\textbf{Answer:} \(P(\text{AB or B | Female}) = (16 + 37) / 430 = 0.123\)

\hypertarget{probability-rules}{%
\chapter{Probability Rules}\label{probability-rules}}

Earlier we computed the conditional probability

\[
\mbox{P(Type O | Female)} = \frac{202}{430} 
\]

The numerator 202 came from the count of Females who have blood Type B, and the denominator 430 is the total number of Females.
If we divide the numerator and denominator by 1000, then the quotient is not changed, so that

\[
\mbox{P(Type O | Female)} = \frac{202/1000}{430/1000} 
= \frac{0.202}{0.43} = \frac{\mbox{P(Type O and Female)}}{\mbox{P(Female)}}
\]

This illustrates a general property of probability: If A and B represent possible outcomes, then the probability of A given B is

\[
\mbox{P(A | B)} = \frac{\mbox{P(A and B)}}{\mbox{P(B)}}
\]

Multiplying on both sides of the above equation by P(B) gives

\[
\mbox{P(A and B)} = \mbox{P(A | B)}\mbox{P(B)}  
\]

This is called the \textbf{multiplication rule}. Reversing the order of A and B above gives

\[
\mbox{P(B and A)} = \mbox{P(B | A)}\mbox{P(A)}  
\]

In the blood type example, it is clear that P(Female and Type O) is the same as P(Type O and Female).
This is true in general: P(A and B) = P(B and A)\}, it follows that
\[
\mbox{Pr(A $|$ B)}\mbox{Pr(B)}  = \mbox{Pr(B $|$ A)}\mbox{Pr(A)} 
\]
so that
\[
\mbox{Pr(A $|$ B)}  = \mbox{Pr(B $|$ A)}\frac{\mbox{Pr(A)}}{\mbox{Pr(B)}}
\]
This formula is called \textbf{Bayes Rule}. Applied to our earlier population, we have
\[
\mbox{Pr(Type O $|$ Female)} = \mbox{Pr(Female $|$ Type O)}\frac{\mbox{Pr(Type O)}}{\mbox{Pr(Female)}}
\]

Two outcomes A and B are \textbf{independent} when \$\mbox{Pr(A $|$ B)} = \mbox{Pr(A)}\$,
which implies that knowing if B has occurred has no effect on the probability of A.

\hypertarget{counterintuitive-applications}{%
\chapter{Counterintuitive Applications}\label{counterintuitive-applications}}

\hypertarget{birthday-paradox}{%
\section{Birthday Paradox}\label{birthday-paradox}}

How many people are needed in a classroom so that the probability of them sharing a birthday is at least \(\frac{1}{2}\)?
Intuitively, one could claim that 183 people in the group would imply that the probability is greater than \(\frac{1}{2}\),
since \(\frac{183}{365} = 0.501\). However, intuition does not correctly solve this problem.

To begin, there are some assumptions that need to be made.
For simplicity, we will ignore leap years and assume that all 365 birthdays have an equal probability of occurring.

We want to compute \(P(\beta)\), the probability that at least 2 people share the same birthday.
Recall from the previous chapter that \(P(\beta) = 1 - P(\beta')\).
In this case, it is much simpler to calculate the probability that no one in the group shares the same birthday with someone else.

Let's start with the simple example involving only two people. The probability that they do not share the same birthday is
\[P(\beta') = \left( \frac{365}{365} \right) \left( \frac{364}{365} \right) = 0.997\]
\[1 - P(\beta') = 0.003\]

Extending this result to a group of five people:
\[P(\beta') = \left( \frac{365}{365} \right) \left( \frac{364}{365} \right) \left( \frac{363}{365} \right) 
\left( \frac{362}{365} \right) \left( \frac{361}{365} \right)\]
\[ = \frac{365!}{365^5(365 - 5)!} = 0.973\]
\[1 - P(\beta') = 0.027\]

This formula can be applied to any number of group size, so that in the general case of \(n\) people,
the probability that at least two share the same birthday is
\[P(\beta) = 1 - \frac{365!}{365^n(365 - n)!}\]

Finally, to solve the original question:
How many people are needed in a classroom so that the probability of them sharing a birthday is at least \(\frac{1}{2}\)?
Iteratively increasing the group size from five, it is clear to see that at a group size of 23, \(P(\beta) > \frac{1}{2}\)
\[P(\beta) = 1 - P(\beta') = 1 - \frac{365!}{365^{23}(365 - 23)!} = 0.507\]

\hypertarget{monty-hall-problem}{%
\section{Monty Hall Problem}\label{monty-hall-problem}}

The original problem was popularized as a letter by Craig F. Whitaker sent to Marilyn vos Savant's column in Parade magazine in 1990.
You can read more about the article here: \url{https://web.archive.org/web/20130121183432/http://marilynvossavant.com/game-show-problem/}

Suppose you're on a game show, and you're given the choice of three doors. Behind one door is a car, behind the others, goats.
You pick a door, say \#1, and the host, who knows what's behind the doors, opens another door, say \#3, which has a goat.
He says to you, ``Do you want to pick door \#2?'' Is it to your advantage to switch your choice of doors?

\hypertarget{graphing-distributions-qualitative-variables}{%
\chapter{Graphing Distributions: Qualitative Variables}\label{graphing-distributions-qualitative-variables}}

Note: Portions below modeled after content from
\emph{Online Statistics Education: A Multimedia Course of Study}
(\url{http://onlinestatbook.com/}) Project Leader: David M. Lane, Rice University

\hypertarget{introduction}{%
\section{Introduction}\label{introduction}}

Suppose that we have a community of 500 people. Each is classified based on
their ABO blood group, which is one of A, B, AB, or O. Below we consider
graphical methods for displaying the results of the blood group classifications.
This starts with tables, and then continues on to how to graph data that
fall into a small number of categories.

This is an example of \emph{qualitative data}. One characteristic of such data is
that the different values do not come with any pre-established ordering.
This can be contrasted with quantitative data, such as the weight of a bag of an
unknown substance, which does have a natural ordering with respect to
different weights.

\hypertarget{frequency-tables}{%
\section{Frequency Tables}\label{frequency-tables}}

All of the graphical methods shown in this section are derived from frequency
tables. Table 1 shows a frequency table for the results of the ABO blood group
classification. It also shows the
relative frequencies, which are the proportion classified in each category.
For example, the relative frequency for group B is 45/500 = 0.09.

\[
\begin{array}{|c|c|c|} \hline
\mathrm{ABO} & & \mathrm{Relative} \\
\mathrm{Group} & \mathrm{Frequency} & \mathrm{Frequency} \\ \hline
\mathrm{A} & 210 & 0.42 \\  \hline
\mathrm{B} & 45  & 0.09 \\  \hline
\mathrm{AB} & 15 & 0.03 \\  \hline
\mathrm{O} & 230 & 0.46 \\  \hline
\end{array}\\
\mbox{Table 1: Frequency Table for ABO Group Data}
\]

\hypertarget{pie-charts}{%
\section{Pie Charts}\label{pie-charts}}

The pie chart in Figure 1 depicts the ABO group data. In a pie chart,
each category is represented by a slice of the pie. The area of the slice is
proportional to the percentage of items in the category -- that is, the
relative frequency multiplied by 100.

\[
\begin{array}{c}
\mbox{Joseph: Please add code to generate pie chart} \\[.1in]
\mbox{Figure 1. Pie chart of relative frequencies for ABO blood groups}
\end{array}
\]

Pie charts are effective for displaying the relative frequencies of a small
number of categories. They are not recommended, however, when you have a large
number of categories. Pie charts can also be confusing when they are used to
compare the outcomes of two different surveys or experiments. In an influential
book on the use of graphs, Edward Tufte asserted, ``The only worse design than
a pie chart is several of them.''

\hypertarget{bar-charts}{%
\section{Bar Charts}\label{bar-charts}}

Bar charts can also be used to represent frequencies of different categories.
A bar chart of the ABO frequencies is shown in Figure 2. Frequencies are shown
on the Y-axis and the blood group is shown on the X-axis.

\[
\begin{array}{c}
\mbox{Joseph: Please add code to generate bar chart} \\[.1in]
\mbox{Figure 2. Bar chart of frequencies for ABO blood groups}
\end{array}
\]

The Y-axis also can show the percentage of observations
instead of the number of observations, as in Figure 3.

\[
\begin{array}{c}
\mbox{Joseph: Please add code to generate bar charts} \\[.1in]
\mbox{Figure 3. Bar chart of percentages for ABO blood groups}
\end{array}
\]

\hypertarget{comparing-distributions}{%
\section{Comparing Distributions}\label{comparing-distributions}}

Often we need to compare different sets of data, or different subsets
within the same overall data set In this case, we are comparing the
``distributions'' of outcomes or responses. Bar charts are
often excellent for illustrating differences between two distributions. Table
2 shows the distribution (in percentages) of ABO blood groups for those in
Albania and Australia.

\[
\begin{array}{|c|c|c|} \hline
\text{ABO} && \\
\text{Group} & \text{Albania} & \text{Australia} \\ \hline
\mathrm{A} & 36.7 & 38.0 \\  \hline
\mathrm{B} & 17.1 & 10.0 \\  \hline
\mathrm{AB} & 6.1 & 3.0 \\  \hline
\mathrm{O} & 40.1 & 49.0 \\  \hline
\end{array}\\
\mbox{Table 2: ABO Blood Group Percentages}
\]
From Table 2 we see that ABO groups B and AB are more common in Albania,
group O is more common in Australia, and group A is similar for both.
This can be seen in the bar chart in Figure 4.

\[
\begin{array}{c}
\mbox{Joseph: Please add code to generate side-by-side bar chart 
(oriented horizontally)} \\[.1in]
\mbox{Figure 4. Bar chart of percentages for ABO blood groups}
\end{array}
\]

The bars in Figure 4 are oriented horizontally rather than vertically. The
horizontal format is useful when you have many categories because there is more
room for the category labels.

\begin{center}\rule{0.5\linewidth}{0.5pt}\end{center}

\hypertarget{exercises-2}{%
\section{Exercises}\label{exercises-2}}

\begin{enumerate}
\def\labelenumi{\arabic{enumi}.}
\tightlist
\item
  \(\text{Put exercises here}\)
\end{enumerate}

\hypertarget{graphing-distributions-histograms}{%
\chapter{Graphing Distributions: Histograms}\label{graphing-distributions-histograms}}

Note: Portions below modeled after content from
\emph{Online Statistics Education: A Multimedia Course of Study}
(\url{http://onlinestatbook.com/}) Project Leader: David M. Lane, Rice University

\hypertarget{introduction-1}{%
\section{Introduction}\label{introduction-1}}

The ``distribution'' of a set of data consists of the set of data values and the
frequency that the values occur.
A \textbf{histogram} is a graphical method for displaying the shape of a distribution.
It is particularly useful when there are a large number of data values, when
it is not practical to list them all.

We begin with an example consisting of the weights of 371 bags containing a
substance suspected of being narcotics. The weights of the bags range from
165 to 313 grams. We begin by creating a frequency table of groupings of scores
as shown in Table 1.

Table 1. Grouped Frequency Distribution of Psychology Test Scores
Interval's Lower Limit Interval's Upper Limit Class Frequency
39.5 49.5 3
49.5 59.5 10
59.5 69.5 53
69.5 79.5 107
79.5 89.5 147
89.5 99.5 130
99.5 109.5 78
109.5 119.5 59
119.5 129.5 36
129.5 139.5 11
139.5 149.5 6
149.5 159.5 1
159.5 169.5 1

To create this table, the range of scores was broken into intervals, called class intervals. The first interval is from 39.5 to 49.5, the second from 49.5 to 59.5, etc. Next, the number of scores falling into each interval was counted to obtain the class frequencies. There are three scores in the first interval, 10 in the second, etc.

Class intervals of width 10 provide enough detail about the distribution to be revealing without making the graph too ``choppy.'' More information on choosing the widths of class intervals is presented later in this section. Placing the limits of the class intervals midway between two numbers (e.g., 49.5) ensures that every score will fall in an interval rather than on the boundary between intervals.

In a histogram, the class frequencies are represented by bars. The height of each bar corresponds to its class frequency. A histogram of these data is shown in Figure 1.

Figure 1. Histogram of scores on a psychology test.

The histogram makes it plain that most of the scores are in the middle of the distribution, with fewer scores in the extremes. You can also see that the distribution is not symmetric: the scores extend to the right farther than they do to the left. The distribution is therefore said to be skewed. (We'll have more to say about shapes of distributions in the chapter " Summarizing Distributions.")

In our example, the observations are whole numbers. Histograms can also be used when the scores are measured on a more continuous scale such as the length of time (in milliseconds) required to perform a task. In this case, there is no need to worry about fence-sitters since they are improbable. (It would be quite a coincidence for a task to require exactly 7 seconds, measured to the nearest thousandth of a second.) We are therefore free to choose whole numbers as boundaries for our class intervals, for example, 4000, 5000, etc. The class frequency is then the number of observations that are greater than or equal to the lower bound, and strictly less than the upper bound. For example, one interval might hold times from 4000 to 4999 milliseconds. Using whole numbers as boundaries avoids a cluttered appearance, and is the practice of many computer programs that create histograms. Note also that some computer programs label the middle of each interval rather than the end points.

Histograms can be based on relative frequencies instead of actual frequencies. Histograms based on relative frequencies show the proportion of scores in each interval rather than the number of scores. In this case, the Y-axis runs from 0 to 1 (or somewhere in between if there are no extreme proportions). You can change a histogram based on frequencies to one based on relative frequencies by (a) dividing each class frequency by the total number of observations, and then (b) plotting the quotients on the Y-axis (labeled as proportion).

There is more to be said about the widths of the class intervals, sometimes called bin widths. Your choice of bin width determines the number of class intervals. This decision, along with the choice of starting point for the first interval, affects the shape of the histogram. There are some ``rules of thumb'' that can help you choose an appropriate width. (But keep in mind that none of the rules is perfect.) Sturges' rule is to set the number of intervals as close as possible to 1 + Log2(N), where Log2(N) is the base 2 log of the number of observations. The formula can also be written as 1 + 3.3 Log10(N), where Log10(N) is the log base 10 of the number of observations. According to Sturges' rule, 1000 observations would be graphed with 11 class intervals since 10 is the closest integer to Log2(1000). We prefer the Rice rule, which is to set the number of intervals to twice the cube root of the number of observations. In the case of 1000 observations, the Rice rule yields 20 intervals instead of the 11 recommended by Sturges' rule. For the psychology test example used above, Sturges' rule recommends 10 intervals while the Rice rule recommends 17. In the end, we compromised and chose 13 intervals for Figure 1 to create a histogram that seemed clearest. The best advice is to experiment with different choices of width, and to choose a histogram according to how well it communicates the shape of the distribution.

\begin{center}\rule{0.5\linewidth}{0.5pt}\end{center}

\hypertarget{exercises-3}{%
\section{Exercises}\label{exercises-3}}

\begin{enumerate}
\def\labelenumi{\arabic{enumi}.}
\tightlist
\item
  \(\text{Put exercises here}\)
\end{enumerate}

  \bibliography{book.bib,packages.bib}

\end{document}
